% Documentation avec latex2man
%
% Auteur  : Gregory DAVID
%%

\documentclass[french]{article}
\usepackage[utf8]{inputenc}
\usepackage[french]{babel}
\usepackage{latex2man}

%% avons-nous le package 'gitinfo' ?
\IfFileExists{gitinfo.sty}{
  \usepackage{gitinfo}
  \setDate{\gitAuthorIsoDate}
  \setVersion{\gitAbbrevHash}
}{
  %%%% sinon on definit la date a la main
  \setDate{\today}
  \setVersion{0}
}
\begin{document}

\begin{Name}{1}{generateurPlaylist}{Grégory DAVID}{Générateur de playlist}{Un générateur de playlist}

  \Prog{generateurPlaylist} est un outil permettant de générer des fichiers de playlist pour les lecteurs multimedia.
  Sa documentation est fournie au moins en page de manuel accessible
  à l'aide de la commande \Cmd{man}{1}.
\end{Name}

\section{Synopsis}
%%%%%%%%%%%%%%%%%%

\Prog{generateurPlaylist} \OptArg{-d}{dureetotale}
                 \OptArg{--genre}{ youpi} \Arg{50}
                 \oOpt{-h}
                 \oOpt{-V}
                 \Arg{sortieM3U}

\Prog{generateurPlaylist}
 \oOpt{-h}
 \oOptArg{--log}{\{DEBUG,INFO,\emph{WARNING},ERROR,CRITICAL\}}
 \oOptArg{--output}{OUTPUT\_FORMAT}

 % --time TIME
 %               [--output OUTPUT OUTPUT] [-g GENRE QUANTITY]
 %               [-s SUB_GENRE QUANTITY] [-b BAND_NAME QUANTITY]
 %               [-a ALBUM_NAME QUANTITY] [-t TRACK_TITLE QUANTITY]
 %               [-G GENRE QUANTITY] [-S SUB_GENRE QUANTITY]
 %               [-B BAND_NAME QUANTITY] [-A ALBUM_NAME QUANTITY]
 %               [-T TRACK_TITLE QUANTITY]

\section{Description}
%%%%%%%%%%%%%%%%%%%%%
\Prog{generateurPlaylist} sert à générer un fichier de playlist \File{sortieM3U} au format M3U afin de blablabla.

La description des différentes options se trouve dans la section suivante.

\section{Options}
%%%%%%%%%%%%%%%%%
% [-h] [--log {DEBUG,INFO,WARNING,ERROR,CRITICAL}] --time TIME
%                [--output OUTPUT OUTPUT] [-g GENRE QUANTITY]
%                [-s SUB_GENRE QUANTITY] [-b BAND_NAME QUANTITY]
%                [-a ALBUM_NAME QUANTITY] [-t TRACK_TITLE QUANTITY]
%                [-G GENRE QUANTITY] [-S SUB_GENRE QUANTITY]
%                [-B BAND_NAME QUANTITY] [-A ALBUM_NAME QUANTITY]
%                [-T TRACK_TITLE QUANTITY]

\begin{Description}\setlength{\itemsep}{0cm}
\item[\OptArg{-d}{dureetotale}] Specifie la durée totale que devra faire la playlist.
\item[\OptArg{--genre}{"formatgenre"}] Permet de définir le ou les genres à prendre en compte. Le formalisme correspond à blablablabla.
\item[\Opt{-h}] Affiche un message d'aide.
\item[\Opt{-V}] Affiche les informations de version.
\end{Description}

\section{Fichiers}
%%%%%%%%%%%%%%%

\begin{Description}\setlength{\itemsep}{0cm}
\item[\File{/usr/local/bin/generateurPlaylist}] Le programme principal.
\item[\File{/etc/generateurPlaylist.cfg}] Le fichier de configuration du programme.
\end{Description}

\section{À voir aussi}
%%%%%%%%%%%%%%%%%%

\Cmd{vlc}{1}, \Cmd{mp3rename}{1}.

\section{Traduction}
%%%%%%%%%%%%%%%%%%%%%%%%%%%%%%%%%%%%%%%%%%%%%%%
L'application a été écrite en langue anglaise, traduite en
français. Nous recherchons des traducteurs dans les langues : English,
German, Italian, Spanish, Russian, Chinese, Japanese.

\section{Pré-requis}
%%%%%%%%%%%%%%%%%%%%%%

\begin{description}\setlength{\itemsep}{0cm}
\item[python] \Prog{generateurPlaylist} a besoin de \Prog{python} version $>=$ 2.7.
\item[scons] Si vous souhaitez installer proprement l'application, il est intéressant
     de disposer du programme \Prog{scons}.
\end{description}

\section{Version}
%%%%%%%%%%%%%%%%%

Version : \gitVtags \Version du \Date.

\section{Licence et Copyright}
%%%%%%%%%%%%%%%%%%%%%%%%%%%%%%%

\begin{description}
\item[Copyright] \copyright\ 2012, Grégory DAVID, 3 rue Beau Soleil, 72700 Allonnes
\item[Licence] Indiquer les termes succins de la licence d'utilisation.

\item[Misc] If you find this software useful, please send me a postcard.
\end{description}

\section{Auteur}
%%%%%%%%%%%%%%%%

\noindent
Grégory DAVID                      \\
3 rue Beau Soleil                       \\
72700 Allonnes                       \\
Email: \Email{gregory.david@ac-nantes.fr}  \\
WWW: \URL{http://www.bts-malraux72.net}.
\LatexManEnd

\end{document}
